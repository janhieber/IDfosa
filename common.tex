

\subsection*{Datenkompression Optimierung}
\begin{itemize}
	\setlength{\parskip}{0pt}
	\setlength{\itemsep}{0pt plus 1pt}
	\item Bestimmte Mindestkompression (Kanälen mit begrenzter Datenrate)
	\item Echtzeit (z.B.\ bei Videokonferenz-Systemen oder digitalen Videorecordern)
	\item Bei verlustbehafteter DK bestimmte Mindestqualität nicht unterschreiten
	\item Maximale Decodierungszeit darf nicht überschritten werden (Videokonferenzsystemen, Bildtelefonen)
\end{itemize}

\subsection*{Quantisierung Optimierung}
\begin{itemize}
	\setlength{\parskip}{0pt}
	\setlength{\itemsep}{0pt plus 1pt}
	\item Minimierung der Quantisiererfehlerleistung (objektives Maß)
	\item Minimierung der subjektiven Wahrnehmbarkeit von Fehlern Bildtelefonen)
\end{itemize}






\begin{multicols}{2}
\subsection*{wahrnehmungsbasierte Codierung}
Perception-based coding

\subsection*{Vektor-Quantisierung}
Aufwand beim Encoder ist deutlich größer als beim Decoder. Gut für Broadcast.

\subsection*{Huffman}
Dekrementieren und mit Einsen auffüllen\\
11 101 100 011 010 0011 0010 0001 0000

\subsection*{Kommazahl zu Binär}
\begin{minipage}{\columnwidth}
In TR mit 2 multiplizieren\\
wenn vor Komma ungerade: 0\\
wenn gerade: 1
\end{minipage}

\subsection*{LZ Algorithmen}
\begin{minipage}{\columnwidth}
\subsubsection*{LZ78}
\begin{itemize}
	\setlength{\parskip}{0pt}
	\setlength{\itemsep}{0pt plus 1pt}
	\item Wörterbuch zu Beginn leer
\end{itemize}
\subsubsection*{LZW}
\begin{itemize}
	\setlength{\parskip}{0pt}
	\setlength{\itemsep}{0pt plus 1pt}
	\item Wörterbuch zu Beginn gefüllt
\end{itemize}
\end{minipage}

\end{multicols}